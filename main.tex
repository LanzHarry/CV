\documentclass[11pt,a4paper]{article}
\usepackage{cv_preamble}
\usepackage{personal_info}

\begin{document}

{\bf \Huge \color{myblue} Harry Lanz - MEng, PhD}

% this section is populated from custom commands in personal_info.sty
% \textbf{Email:} \email
% \hfill \textbf{LinkedIn:} \linkedin\\
% \textbf{Tel:} \tel
% \hfill \textbf{Github:} \github

\begin{tblr}{
  colspec = {l l X r r},
  width = \linewidth,
  rows = {rowsep=0pt},
  leftsep = 0pt,
  rightsep = 0pt
}
\textbf{Email:}    & \email & & \textbf{LinkedIn:} & \linkedin \\
\textbf{Tel:}      & \tel   & & \textbf{GitHub:}   & \github
\end{tblr}

\vspace{0.2cm}\hrule

{\bf Junior Software Engineer with experience in \cpp and Linux systems development, building performance-critical, end-to-end software for real-time data acquisition, hardware control, and large-scale data processing. Strong grounding in networking fundamentals (TCP/UDP), asynchronous and concurrent programming, and high performance computing on Linux.}\\ \vspace{-0.2cm} \hrule


\section*{Experience}
\hrule

\textbf{PhD:}

\begin{itemize}
  \item Designed and maintained end-to-end, performance-critical software systems in \cpp and Python for a custom MRI platform, operating under real-time and hardware-safety constraints.
  \item Implemented \cpp control software for motion and acquisition hardware using CANbus, including message parsing, state machines, error handling, and logging.
  \item Built multi-stage data pipelines handling large binary datasets, with careful attention to memory usage, throughput, and reproducibility.
  \item Built interactive visualisation and segmentation tools to support non-technical users and downstream analysis.
\end{itemize}

\textbf{Master's project:}

\begin{itemize}
  \item Extended a large Fortran-based CFD codebase on Linux to implement new numerical methods for identifying chaotic regions in turbulent flow.
  \item Worked within an existing production codebase, adding functionality without breaking established workflows.
\end{itemize}

\textbf{Aeronautics high performance computing:}

\begin{itemize}
  \item Implemented a parallel \cpp fluid dynamics solver using MPI, focusing on data decomposition, inter-process communication, and performance scaling on Linux HPC clusters.
\end{itemize}

\textbf{Personal Projects:}

\begin{itemize}
  \item Designed embedded \cpp applications on ESP32 involving networked device control over TCP/IP and REST interfaces; implemented request handling, state management, and failure recovery.
  \item Built and operated Linux services on Raspberry Pi including VPN, DNS, and media servers, deployed using Docker and Docker Compose; configured networking, port forwarding, and service isolation.
\end{itemize}
  

\section*{Education}
\hrule

\textbf{PhD Mechanical Engineering}, Imperial College London \hfill [2020--2025]\\PhD awarded; viva passed November 2025

\textbf{MEng Aeronautics} (Integrated Masters), Imperial College London \hfill [2015--2020]\\First class honours; Dean's List (Master's year)

\newpage 

\section*{Skills}
\hrule

\textbf{Programming:} Modern \cpp, Python, Go, SQL, Bash, OOP

\textbf{Systems \& networking:} Linux (user-space systems programming), TCP/IP, UDP, sockets, asynchronous programming, REST APIs, Docker, basic networking concepts (DNS, VPNs)

\textbf{Developer tooling:} Git, CI/CD (GitHub Actions), VS Code, Neovim, pytest

\textbf{Data \& scientific computing:} NumPy, SciPy, PyTorch, MATLAB, Fortran

\section*{Selected research output}
\hrule

\textbf{First In Vivo Magic Angle Directional Imaging Using Dedicated Low-Field MRI} \hfill [2025]\\
\textit{Published in the Journal of Magnetic Resonance in Medicine:} \href{https://doi.org/10.1002/mrm.30332}{doi.org/10.1002/mrm.30332}

\begin{itemize}
  \item Built and maintained a Python and \cpp software stack enabling data processing, visualisation, and analysis by non-technical users.
\end{itemize}

\section*{Other experience}
\hrule

\begin{itemize}
  \item Supplementary software engineering training (Boot.dev - Python, Go, Linux, SQL, Docker).
  \item Supervised Master's students on Python-based image reconstruction and performed code reviews.
  \item Programmatic CAD modelling for 3D printing using Python.
\end{itemize}


References are available upon request.


\end{document}