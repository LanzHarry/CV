\documentclass[11pt,a4paper]{article}
\usepackage{cv_preamble}
\usepackage{personal_info}

\begin{document}

{\bf \Huge \color{myblue} Harry Lanz - MEng, PhD}

% this section is populated from custom commands in personal_info.sty
% \textbf{Email:} \email
% \hfill \textbf{LinkedIn:} \linkedin\\
% \textbf{Tel:} \tel
% \hfill \textbf{Github:} \github

\begin{tblr}{
  colspec = {l l X r r},
  width = \linewidth,
  rows = {rowsep=0pt},
  leftsep = 0pt,
  rightsep = 0pt
}
\textbf{Email:}    & \email & & \textbf{LinkedIn:} & \linkedin \\
\textbf{Tel:}      & \tel   & & \textbf{GitHub:}   & \github
\end{tblr}

\vspace{0.2cm}\hrule
{\bf Junior Software Engineer (backend / full stack) with experience building production-grade data pipelines, optimisation algorithms, and hardware-control software. Strong engineering fundamentals, comfortable working across the stack in Python and \cpp, and experienced with Linux-based development, git workflows, SQL, and Docker.} \\ \vspace{-0.2cm} \hrule

\section*{Experience}
\hrule

\textbf{PhD:}

\begin{itemize}
  \item Designed and maintained end-to-end software systems in Python and \cpp supporting a custom MRI system, spanning data acquisition, processing, visualisation, and hardware control.
  \item Developed data processing and image analysis pipelines in Python (NumPy, SciPy, PyTorch) and Matlab, handling large 3D imaging datasets.
  \item Implemented hardware control and GUI software in \cpp to coordinate scanner motion via CANbus, including logging and safety checks.
  \item Built interactive visualisation and segmentation tools to support non-technical users and downstream analysis.

\end{itemize}

\textbf{Master's project:}

\begin{itemize}
  \item Extended a large Fortran-based CFD codebase to support new numerical analysis methods for identifying chaotic turbulent flow regions.
  \item Worked within an existing production codebase, adding functionality without breaking established workflows.
\end{itemize}

\textbf{Aeronautics high performance computing:}

\begin{itemize}
  \item Implemented a parallel \cpp solver for fluid dynamics simulations using domain decomposition and deployed it on a shared HPC cluster environment using MPI.
\end{itemize}

\textbf{Mindbridge:}

\begin{itemize}
  \item Developed a Python-based data ingestion tool to digitise handwritten questionnaire responses from scanned documents into a CSV.
  \item Built a PyQt GUI workflow enabling human validation of ambiguous entries before export to CSV for downstream analysis.
\end{itemize}
  

\section*{Education}
\hrule

\textbf{PhD Mechanical Engineering}, Imperial College London \hfill [2020--2025]\\PhD awarded; viva passed November 2025

\textbf{MEng Aeronautics} (Integrated Masters), Imperial College London \hfill [2015--2020]\\First class honours; Dean's List (Master's year)

\newpage 

\section*{Skills}
\hrule

\textbf{Programming languages:} Python, \cpp, Go, SQL, Bash

\textbf{Backend \& systems:} Linux, Docker, REST APIs and basic networking

\textbf{Developer tooling:} Git, GitHub, CI/CD (GitHub Actions), VS Code, Neovim, pytest

\textbf{Data \& scientific computing:} NumPy, SciPy, PyTorch, MATLAB, Fortran

\textbf{GUI \& desktop:} Qt, PyQt

\section*{Selected research output}
\hrule

\textbf{First In Vivo Magic Angle Directional Imaging Using Dedicated Low-Field MRI} \hfill [2025]\\
\textit{Published in the Journal of Magnetic Resonance in Medicine:} \href{https://doi.org/10.1002/mrm.30332}{doi.org/10.1002/mrm.30332}

\begin{itemize}
  \item Built and maintained a Python and \cpp software stack enabling data processing, visualisation, and analysis by non-technical users.
\end{itemize}

\section*{Other experience}
\hrule

\begin{itemize}
  \item Supplementary software engineering training (Boot.dev - Python, Go, Linux, SQL, Docker)
  \item Supervised Master's students on Python-based image reconstruction and performed code reviews
  \item Programmatic CAD modelling for 3D printing using Python.
\end{itemize}


References are available upon request.



\end{document}